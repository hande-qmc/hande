\documentclass[a4paper, 11pt]{revtex4}

\usepackage{amsmath}

\newcommand{\bi}{\mathbf{i}}
\newcommand{\bj}{\mathbf{j}}
\newcommand{\bz}{\mathbf{0}}
\newcommand{\br}{\mathbf{r}}
\newcommand{\bra}{\langle}
\newcommand{\ket}{\rangle}

\begin{document}

\title{Real-space components of Determinants of Molecular Orbitals}

\maketitle

Let $\psi_i$ be a local spin-orbital, where index $i$ comprises both a spatial index (a site index in the case of the Hubbard model) and a spin index and let $\phi_i$ be a molecular spin-orbital, for instance obtained from a Hartree--Fock (HF) calculation, where index $i$ might comprise of a Bloch wavevector which is in turn defined by 3 integers, a band index and a spin index.

The set $\{\psi_i\}$ and the set $\{\phi_i\}$ span the same space and so there exists a matrix of coefficients, $\boldsymbol{\alpha}$, such that
\begin{equation}
\psi_i = \sum_j \alpha_{ji} \phi_i.
\end{equation}
The coefficients are known from an HF calculation or perhaps even from symmetry alone in the case of the one-band Hubbard model.

[JSS: Perhaps the above might be more clearly represented using covariant and contravariant notation?  This might be a very (hardcode) quantum chemist thing to do though...]

The HF determinant, $D_0$, is constructed using the $N$ lowest energy spin-orbitals, $\psi_1, \psi_2, \cdots, \psi_N$:
\begin{equation}
D_0(x_1, x_2, \cdots, x_N) = \frac{1}{\sqrt{N!}}
\begin{vmatrix}
\psi_1(x_1) & \psi_1(x_2) & \cdots & \psi_1(x_N) \\
\psi_2(x_1) & \psi_2(x_2) & \cdots & \psi_2(x_N) \\
\vdots      & \vdots      & \ddots & \vdots      \\
\psi_N(x_1) & \psi_N(x_2) & \cdots & \psi_N(x_N)
\end{vmatrix}
\end{equation}
where $x_i=(\br,\sigma_i)$ represents both the position and spin components of electron $i$.

$D_0$ can obviously be expanded in the basis of determinants of lcoal orbitals:
\begin{equation}
D_0(x_1, x_2, \cdots, x_N) = \sum_{i_1<i_2<\cdots<i_N} C_{i_1i_2\cdots i_N} \frac{1}{\sqrt{N!}}
\begin{vmatrix}
\phi_{i_1}(x_1) & \phi_{i_1}(x_2) & \cdots & \phi_{i_1}(x_N) \\
\phi_{i_2}(x_1) & \phi_{i_2}(x_2) & \cdots & \phi_{i_2}(x_N) \\
\vdots      & \vdots      & \ddots & \vdots      \\
\phi_{i_N}(x_1) & \phi_{i_N}(x_2) & \cdots & \phi_{i_N}(x_N)
\end{vmatrix}.
\end{equation}
We need to find the coefficients, $C_{i_1i_2\cdots i_N}$ for all index sets $\{i_1i_2\cdots i_N|i_1<i_2<\cdots<i_N\}$.

First, we must show that the determinants of local orbitals are orthonormal.  Let
\begin{align}
D_{i_1i_2\cdots i_N} &= \frac{1}{\sqrt{N!}}
\begin{vmatrix}
\phi_{i_1}(x_1) & \phi_{i_1}(x_2) & \cdots & \phi_{i_1}(x_N) \\
\phi_{i_2}(x_1) & \phi_{i_2}(x_2) & \cdots & \phi_{i_2}(x_N) \\
\vdots      & \vdots      & \ddots & \vdots      \\
\phi_{i_N}(x_1) & \phi_{i_N}(x_2) & \cdots & \phi_{i_N}(x_N)
\end{vmatrix} \\
&= \frac{1}{\sqrt{N!}} \sum_P (-1)^{\xi_P} \phi_{i_{P1}}(x_1) \phi_{i_{P2}}(x_2) \cdots \phi_{i_{PN}}(x_N)
\end{align}
where $P=(P1,P2,\cdots,PN)$ is some permutation of the indices $(1,2,\cdots,N)$ and $\xi_P$ is the parity of the permutation $P$.  Then the overlap between two determinants of local orbitals is given by
\begin{equation}
\begin{split}
\bra D_{j_1j_2\cdots j_N} | D_{i_1i_2\cdots i_N} \ket = 
\frac{1}{\sqrt{N!}}& \sum_P (-1)^{\xi_P}
\frac{1}{\sqrt{N!}} \sum_Q (-1)^{\xi_Q} \\
&\int
\phi_{j_{P1}}^*(x_1) \phi_{j_{P2}}^*(x_2) \cdots \phi_{j_{PN}}^*(x_N) \\
&\quad\ \phi_{i_{Q1}}(x_1) \phi_{i_{Q2}}(x_2) \cdots \phi_{i_{QN}}(x_N)
dx_1 dx_2 \cdots dx_N.
\end{split}
\end{equation}

Now, whatever the permutation $Q$, we can write it as the combination of two other permutations:
\begin{gather}
Q = PR \\
\xi_Q = \xi_P + \xi_R
\end{gather}
where $R$ is some other permutation.  Thus
\begin{equation}
\sum_Q (-1)^{\xi_Q} \phi_{i_{Q1}}(x_1) \phi_{i_{Q2}}(x_2) \cdots \phi_{i_{QN}}(x_N) = \sum_R 
(-1)^{\xi_P + \xi_R} \phi_{i_{PR1}}(x_1) \phi_{i_{PR2}}(x_2) \cdots \phi_{i_{PRN}}(x_N) 
\end{equation}
and so 
\begin{equation}
\begin{split}
\bra D_{j_1j_2\cdots j_N} | D_{i_1i_2\cdots i_N} \ket = \frac{1}{N!} \sum_P \sum_R (-1)^{\xi_R}
& \int 
\phi_{j_{P1}}^*(x_1) \phi_{j_{P2}}^*(x_2) \cdots \phi_{j_{PN}}^*(x_N) \\
& \quad\ \phi_{i_{PR1}}(x_1) \phi_{i_{PR2}}(x_2) \cdots \phi_{i_{PRN}}(x_N) 
dx_1 dx_2 \cdots dx_N.
\end{split}
\end{equation}
All permutations $P$ now give identical contributions as changing $P$ just reorders terms in the integrand.  Hence the sum over $P$ exactly cancels with the $N!^{-1}$ normalisation factor and we obtain:
\begin{align}
\begin{split}
\bra D_{j_1j_2\cdots j_N} | D_{i_1i_2\cdots i_N} \ket =&
\sum_R (-1)^{\xi_R} \int
\phi_{j_{1}}^*(x_1) \phi_{j_{2}}^*(x_2) \cdots \phi_{j_{N}}^*(x_N) \\
&\hphantom{\sum_R (-1)^{\xi_R}\int} \phi_{i_{R1}}(x_1) \phi_{i_{R2}}(x_2) \cdots \phi_{i_{RN}}(x_N) 
dx_1 dx_2 \cdots dx_N
\end{split} \\
\begin{split}
=&
\sum_R (-1)^{\xi_R} 
\int \phi_{j_{1}}(x_1)^* \phi_{i_{R1}}(x_1) dx_1 \\
& \hphantom{\sum_R (-1)^{\xi_R}}
\int \phi_{j_{2}}(x_2)^* \phi_{i_{R2}}(x_2) dx_2 \\
& \hphantom{\sum_R (-1)^{\xi_R}}
\cdots \\
& \hphantom{\sum_R (-1)^{\xi_R}}
\int \phi_{j_{N}}(x_N)^* \phi_{i_{RN}}(x_N) dx_N 
\end{split} \\
\begin{split}
=& \sum_R (-1)^{\xi_R} 
\delta_{j_1,i_{R1}}
\delta_{j_2,i_{R2}}
\cdots
\delta_{j_N,i_{RN}}
\end{split}
\end{align}
assuming that the local orbitals are sufficiently localised such that they form an orthonormal set.  This vanishes unless $j_1=i_{R1}, j_2=i_{R2}, \cdots, j_N=i_{RN}$.  Since we defined $i_1<i_2<\cdots<i_N$ and $j_1<j_2<\cdots<j_N$, this is only possible when $R$ is the identity permutation and $j_1=i_1, j_2=i_2, \cdots, j_N=i_N$.  Thus
\begin{equation}
\bra D_{j_1j_2\cdots j_N} | D_{i_1i_2\cdots i_N} \ket = \delta_{j_1i_1} \delta_{j_2i_2} \cdots \delta{j_Ni_N}
\end{equation}
as required.

We now use this property to find the overlap of a determinant of local orbitals with the Hartree--Fock determinant.  Since the determinants of local orbitals, $D_{i_1,i_2,\cdots,i_N}$ form an orthonormal basis, and since
\begin{equation}
D_0(x_1, x_2, \cdots, x_N) = \sum_{i_1<i_2<\cdots<i_N} C_{i_1i_2\cdots i_N} D_{i_1,i_2,\cdots,i_N}
\end{equation}
it follows that
\begin{equation}
C_{i_1i_2\cdots i_N} = \bra D_{i_1,i_2,\cdots,i_N} | D_0(x_1, x_2, \cdots, x_N) \ket.
\end{equation}
We can evaluate this in a similar fashion:
\begin{align}
\begin{split}
C_{i_1i_2\cdots i_N} =& \frac{1}{N!} \sum_P (-1)^{\xi_P} \sum_Q (-1)^{\xi_Q} \\
& \quad \int
\phi_{j_{P1}}^*(x_1) \phi_{j_{P2}}^*(x_2) \cdots \phi_{j_{PN}}^*(x_N)
\psi_{i_{Q1}}(x_1) \psi_{i_{Q2}}(x_2) \cdots \psi_{i_{QN}}(x_N)
dx_1 dx_2 \cdots dx_N \end{split} \\
                     =& \sum_R (-1)^{\xi_R} \int 
\phi_{i_{1}}^*(x_1) \phi_{i_{2}}^*(x_2) \cdots \phi_{i_{N}}^*(x_N)
\psi_{j_{R1}}(x_1) \psi_{j_{R2}}(x_2) \cdots \psi_{j_{RN}}(x_N)
dx_1 dx_2 \cdots dx_N.
\end{align}
Now as
\begin{equation}
\int \phi_{i_1}^*(x_1) \psi_{j_{R1}}(x_1) dx_1 = \alpha_{i_1 j_{R1}}
\end{equation}
by construction, it follows that $C_{i_1i_2\cdots i_N}$ can be found by taking the determinant of the appropriate elements of the $\boldsymbol{\alpha}$ matrix:
\begin{align}
C_{i_1i_2\cdots i_N} &= \sum_R (-1)^{\xi_R} \alpha_{i_1 j_{R1}} \alpha_{i_2 j_{R2}} \cdots \alpha_{i_N j_{RN}} \\
                     &= 
\begin{vmatrix}
\alpha_{i_1 j_1} & \alpha_{i_1 j_2} & \cdots & \alpha_{i_1 j_N} \\
\alpha_{i_2 j_1} & \alpha_{i_2 j_2} & \cdots & \alpha_{i_2 j_N} \\
\vdots           & \vdots           & \ddots & \vdots           \\
\alpha_{i_N j_1} & \alpha_{i_N j_2} & \cdots & \alpha_{i_N j_N}
\end{vmatrix}.
\end{align}

\end{document}
